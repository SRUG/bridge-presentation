\documentclass[16pt]{beamer}
\usepackage[utf8]{inputenc}
\usepackage[T1]{fontenc}
\usepackage{graphicx}
\usepackage[polish]{babel}
\usepackage{polski}
\input{pygments}

\usetheme{Pittsburgh}
\usenavigationsymbolstemplate{} % turn off navigation icons
\setbeamercovered{transparent}

\author{Śląska Grupa Użytkowników języka Ruby\\
  \footnotesize{Jakub Kuźma}}
\title{Podstawy tworzenia aplikacji internetowych z wykorzystaniem YUI 3 i Rails 3}

\begin{document}

\frame{\titlepage}

\begin{frame}
  \frametitle{Wstęp}
  \begin{center}
    Bogata aplikacja internetowa?
  \end{center}
\end{frame}

\begin{frame}
  \frametitle{Brydż - licytacja}
  \begin{figure}
    \includegraphics[width=\linewidth]{bbo-auction.png}
  \end{figure}
\end{frame}

\begin{frame}
  \frametitle{Brydż - rozgrywka}
  \begin{figure}
    \includegraphics[width=\linewidth]{bbo-play.png}
  \end{figure}
\end{frame}

\begin{frame}
  \frametitle{Wymagania}
  \begin{itemize}
  \item komunikacja z serwerem bez przeładowania strony
  \item walidacja działań użytkownika przed wysłaniem żądania
  \item rozbicie interfejsu na małe, niezależne elementy
  \end{itemize}
\end{frame}

\begin{frame}
  \frametitle{jQuery}
  \begin{center}
    absurdalnie prosty AJAX, obsługa zdarzeń, manipulacja DOM
  \end{center}
\end{frame}

\begin{frame}
  \frametitle{jQuery UI}
  \begin{center}
    jQuery UI jest tym dla RIA, czym jQuery było dla AJAX
  \end{center}
\end{frame}

\begin{frame}
  \frametitle{YUI}
  \begin{figure}
    \includegraphics[width=0.8\linewidth]{yui.jpg}
  \end{figure}
\end{frame}

\begin{frame}
  \frametitle{YUI 3 - w rolach głównych}
  \begin{itemize}
  \item Developer Tools
  \item Core
  \item Utilities
  \item Component Infrastructure
  \item Widgets
  \item Plugins
  \end{itemize}
\end{frame}

\begin{frame}
  \frametitle{Component Infrastructure}
  \begin{itemize}
  \item Attribute
  \item Base
  \item Widget
  \item Plugin
  \end{itemize}
\end{frame}

\begin{frame}
  \frametitle{Attribute}
  \begin{itemize}
  \item settery, gettery, walidatory
  \item zdarzenie attributeChanged
  \item klonowanie obiektów
  \item obsługa zagnieżdżonych wartości (np. strings.messages.hello)
  \end{itemize}
\end{frame}

\begin{frame}[fragile]
  \frametitle{Attribute - przykład}
  \input{attribute}
\end{frame}

\begin{frame}[fragile]
  \frametitle{Attribute - użycie}
  \input{attribute-usage}
\end{frame}

\begin{frame}
  \frametitle{Base}
  \begin{itemize}
  \item konstruktor i destruktor
  \item atrybuty
  \item obsługa zdarzeń
  \item obsługa rozszerzeń i wtyczek
  \end{itemize}
\end{frame}

\begin{frame}[fragile]
  \frametitle{Base - przykład}
  \input{base}
\end{frame}

\begin{frame}
  \frametitle{Widget}
  \begin{itemize}
  \item HTML Parser
  \item renderUI
  \item bindUI
  \item syncUI
  \item boundingBox, contentBox
  \end{itemize}
\end{frame}

\begin{frame}[fragile]
  \frametitle{Widget - przykład}
  \begin{footnotesize}
    \input{widget}
  \end{footnotesize}
\end{frame}

\begin{frame}
  \frametitle{Plugin}
  \begin{itemize}
  \item nieinwazyjne ,,wzbogacanie'' obiektów
  \item afterHostEvent, beforeHostMethod, itp.
  \end{itemize}
\end{frame}

\begin{frame}[fragile]
  \frametitle{Plugin - przykład}
  \input{plugin}
\end{frame}

\begin{frame}
  \frametitle{Custom Events}
  \begin{itemize}
  \item niezależne od DOM
  \item to samo API (bubbling - stopPropagation, defaultFn - preventDefault)
  \end{itemize}
\end{frame}

\begin{frame}
  \frametitle{Inne}
  \begin{itemize}
  \item operacje na tablicach (every, find, reduce, itp.)
  \item OOP (augment, extend, bind, rbind)
  \item YUI 3 Gallery
  \end{itemize}
\end{frame}

\begin{frame}
  \frametitle{Rails 3}
  \begin{itemize}
  \item logika gry - walidacja
  \item tradycyjny serwis webowy oraz API % respondery - zmiana pollingu na websockety
  \end{itemize}
\end{frame}

\begin{frame}
  \frametitle{node.js}
  \begin{itemize}
  \item YUI3 to CommonJS
  \item podmiana modułu Get
  \item obsługa DOM dzięki jsdom
  \end{itemize}
\end{frame}

\begin{frame}
  \frametitle{Warto zwrócić uwagę}
  \begin{itemize}
  \item underscore.js
  \item mustache.js
  \end{itemize}
\end{frame}

% slajd z Drogusem? :-)

\end{document}
